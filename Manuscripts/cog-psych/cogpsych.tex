%!TEX output_directory = latex_out/

\documentclass[12pt]{article}
\usepackage[letterpaper, margin=1in, headheight=15pt]{geometry}
\usepackage{amsmath}
\usepackage{setspace}
\usepackage{pgfplots}
\usepackage{fancyhdr}
\usepackage{fancyvrb}
\usepackage[natbibapa]{apacite}

% set up PGF
\pgfplotsset{compat=1.6}

% set up header % 
\pagestyle{fancy}
\fancyhf{} % sets both header and footer to nothing
\renewcommand{\headrulewidth}{0pt}
\lhead{RUNNING HEAD: Contrast in Concept Generation}
\fancyhead[R]{\thepage}


% author info:
% https://www.elsevier.com/journals/cognitive-psychology/0010-0285/guide-for-authors


\begin{document}



% ------- TITLE PAGE ------- %
\begin{center}
\hfill
\\[1in]

The Role of Contrast in Concept Generation.
% ---- other ideas:
% - The Role of Contrast in Concept Generation.
% - Creating Something Different: The Role of Contrast in Concept Generation.
% - Something New: The Role of Contrast in Concept Generation.
% - What's New In Concept Generation? The Role of Category Contrast.


\vfill

Nolan Conaway\textsuperscript{1}, 
Kenneth J. Kurtz\textsuperscript{2}, 
\& Joseph L. Austerweil\textsuperscript{1}
\\[\baselineskip]
\textsuperscript{1}University of Wisconsin-Madison, Department of Psychology, Madison, WI, USA
\textsuperscript{2}Department of Psychology, Binghamton University, Binghamton, NY, USA
\\[1in]

\vfill

Author Note


Correspondence concerning this article should be addressed to: 
Joseph Austerweil, 1202 West Johnson Street, Madison, WI 53706.
E-mail: austerweil@wisc.edu

\end{center}
\clearpage


% ------- ABSTRACT PAGE ------- %
\doublespacing
\section{Abstract}

Filler!

\setlength\parindent{0.5in}
\textit{Keywords}: categorization, concepts, creativity, generation
\clearpage


% ------- BEGIN! ------- %
\begin{flushleft}

\section{Introduction}
\setlength\parindent{0.5in}




\section{Experiment 1}



\end{flushleft}


% references section
\clearpage
\setlength{\bibsep}{10pt}
\setstretch{1}
\bibliography{references.bib}
\clearpage



\end{document}
