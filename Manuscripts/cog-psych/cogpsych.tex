%!TEX output_directory = latex_out/

\documentclass[12pt]{article}
\usepackage[letterpaper, margin=1in, headheight=15pt]{geometry}
\usepackage{amsmath}
\usepackage{setspace}
\usepackage{pgfplots}
\usepackage{fancyhdr}
\usepackage{fancyvrb}
\usepackage{csquotes}
\usepackage[natbibapa]{apacite}

% set up PGF
\pgfplotsset{compat=1.6}
\newcommand\inputpgf[2]{{
\let\pgfimageWithoutPath\pgfimage
\renewcommand{\pgfimage}[2][]{\pgfimageWithoutPath[##1]{#1/##2}}
\input{#1/#2}
}}

% set up header % 
\pagestyle{fancy}
\fancyhf{} % sets both header and footer to nothing
\renewcommand{\headrulewidth}{0pt}
\lhead{RUNNING HEAD: Similarity and Contrast in Concept Generation}
\fancyhead[R]{\thepage}


% author info:
% https://www.elsevier.com/journals/cognitive-psychology/0010-0285/guide-for-authors


\begin{document}



% ------- TITLE PAGE ------- %
\begin{center}
\hfill
\\[1in]

Creating Something Different: Similarity and Contrast Effects in Concept Generation.
% ---- other ideas:
% - The Role of Contrast in Concept Generation.
% - Creating Something Different: The Role of Contrast in Concept Generation.
% - Something New: The Role of Contrast in Concept Generation.
% - What's New In Concept Generation? The Role of Category Contrast.


\vfill

Nolan Conaway\textsuperscript{1}, 
Kenneth J. Kurtz\textsuperscript{2}, 
\& Joseph L. Austerweil\textsuperscript{1}
\\[\baselineskip]
\textsuperscript{1}University of Wisconsin-Madison, Department of Psychology, Madison, WI, USA
\textsuperscript{2}Department of Psychology, Binghamton University, Binghamton, NY, USA
\\[1in]

\vfill

Author Note


Correspondence concerning this article should be addressed to: 
Joseph Austerweil, 1202 West Johnson Street, Madison, WI 53706.
E-mail: austerweil@wisc.edu

\end{center}
\clearpage


% ------- ABSTRACT PAGE ------- %
\doublespacing
\section{Abstract}

Filler!

\setlength\parindent{0.5in}
\textit{Keywords}: categorization, concepts, creativity, generation
\clearpage


% ------- BEGIN! ------- %
\begin{flushleft}

\section{Introduction}
\setlength\parindent{0.5in}


The creation of novel concepts and ideas is a highly intriguing -- yet infrequently studied -- topic of research in human cognition. The creative use of conceptual knowledge is a core element of scientific investigation, wherein the generation of new ideas is critical to designing experiments and explaining observations. However, due to its complexity, little is presently known about the cognitive processes underlying the creative generation of new concepts. 

Much of what we know about conceptual generation comes from the foundational literature on creative cognition. In a classic series of reports, Ward \& colleagues \citep{ward1995s,ward1994structured,marsh1999inadvertent,ward2002role,smith1993constraining} established that category generation is highly constrained by prior knowledge: generated categories tend to consist of features observed in known categories, and they tend to exhibit the distributional properties as found in known categories. In a classic study, \cite{ward1994structured} asked participants to generate new species of alien animals by drawing and describing members of the species. People tended to generate species with the same features as on Earth (e.g., eyes, legs, wings), and possessing the same feature correlations as on Earth (e.g., feathers co-occur with wings). Likewise, aliens drawn from the same species tended to share more features with one another compared to members of opposite species. 

% [[NBC]] Explain copy and tweak here?

Much of the work from this area \citep[e.g.,][]{smith1993constraining,marsh1999inadvertent} focuses on how sample cues (such as an example of a species generated by other participants) can drastically diminish creativity. However, the broader set of observations made by Ward \& colleagues provide a great deal of insight into the nature of creative generation. They indicate that people rely strongly on prior knowledge in the creative process, and people generate concepts in accord with what they already know. Problematically, however, the experimental paradigms employed in these studies were relatively uncontrolled. Specifically, in order to allow for maximum creative freedom, participants were not constrained as to what they could generate. While these procedures do ensure creative freedom, they render difficult the development of formal theories to explain conceptual generation.

\cite{jern2013probabilistic} recently showed that creative generation could be studied in a more controlled manner through the well-developed methods of an artificial categorization paradigm \citep[see][]{kurtz2015human}. In their experiments 3 and 4, participants were exposed to members of experimenter-defined categories of "crystals" varying in size, hue, and saturation. Following a training phase during which the experimenter-defined categories were learned, participants were asked to generate novel categories of crystals. In a finding mirroring that of the \cite{ward1994structured} studies, \citeauthor{jern2013probabilistic} found that participants generated categories with the same distributional properties as the experimenter-defined categories: for example, after training on categories with a positive correlation between the size and saturation features (larger sized crystals were more saturated), participants generated novel categories with a positive correlation.

The \cite{jern2013probabilistic} report thus represents an important step towards understanding the cognitive processes underlying creative generation: not only did they find that generation in an artificial domain resembles the more naturalistic cases observed by Ward \& colleagues, but they also found that theories of conceptual generation could be evaluated more formally through comparison to the participant-generated categories.

% [[NBC]] explain their models here...

It is worth noting, however, that the literature reviewed above provides only a limited view of category generation: the main focus of the published experiments has been on the distributional correspondences between learned and generated categories. In this paper, we seek to identify other important constraints on the processes involved in creative generation. We also introduce a novel exemplar-based model of category generation, PACKER (\textit{Producing Alike and Contrasting Knowledge using Exemplar Representations}), which explains much of what we observe in the creative use of conceptual knowledge. In the sections below, we introduce the PACKER model and its core theoretical claims.

\subsection{Something Different: A Role For Contrast}

In order to generate a novel concept, individuals must produce something that is in some capacity \textit{different} from what they already know. Thus, in a trivial sense, contrast can be viewed as a fundamental constraint on creative generation: new concepts must firstly be different from existing ones. However the role of contrast in category generation has not been systematically evaluated: how do people generate concepts such that the product is (A) different from what is already known, and (B) internally coherent?


\clearpage

\section{The PACKER Model}

The PACKER model is an extension of the influential Generalized Context Model of category learning \citep[GCM;][]{nosofsky1984choice}. It assumes that each category is encoded by a set of exemplars within a $k$-dimensional psychological space, and that generation is constrained by both similarity to members of the target category (the category in which a stimulus is being generated) as well as similarity to members of other categories. 

As in the GCM, the similarity between two examples, $s\left(x_i, x_j\right)$, is an inverse-exponential function of distance:
% 
\begin{equation}
  s\left(x_i,x_j\right) = \exp \left\{ -c \sum_{k}{ \left| x_{ik} - x_{jk} \right|}w_k \right\}
  \label{eq:similarity}
\end{equation}
% 
where $w_k$ is the attention weighting of dimension $k$ ($w_k \geq 0$ and $\sum_k{w_k} = 1$), accounting for the relative importance of each dimension in similarity calculations, and $c$ ($c>0$) is a specificity parameter controlling the spread of exemplar generalization. For simplicity, our simulations will use uniform attention weights, except in our discussion of individual differences.  

To generate a new example, the model considers both the similarity to examples from contrast categories as well as the similarity to examples (if any exist) in the target category. The aggregated similarity $a$ between generation candidate $y$ and stored exemplars $x$ is given by:
% 
\begin{equation}
    a(y, x) = \sum_j{f(x_j) s(y, x_j)}
\end{equation}
% 
where $f(x_j)$ is a function specifying the extent to the exemplar contributes to generation. PACKER sets $f(x_j)$ depending on exemplar $x_j$'s category membership: $f(x_j) = \phi$ if $x_j$ is a member of a contrast category, and $f(x_j) = \gamma$ if $x_j$ is a member of the target category. $\phi$ and $\gamma$ are free parameters ($-\infty \leq \phi, \gamma \leq \infty$) controlling the contribution of contrast- and target-category similarity, respectively. Larger absolute values result in greater consideration of those exemplars, with values of 0 eliminating their effect. A negative value for $f(x_j)$ produces a `repelling' effect (exemplars are less likely to be generated nearby $x_j$). Conversely, a positive value for $f(x_j)$ produces an `attracting' effect (exemplars are more likely to be generated nearby $x_j$). 

PACKER's core proposal is that new categories should be different from existing categories, and same-category exemplars should be similar to one another. This is realized when $\phi \leq 0$, and $\gamma \geq 0$. Negative $\phi$ values encourage $y$ to be distant from the contrast category (as similarity to contrast category exemplars are subtracted during aggregation). Positive $\gamma$ values encourage $y$ to be close to other exemplars of the target category. When $|\phi| = \gamma$, the repulsion effect from contrast categories is equal to the attraction effect to the target category.

The probability that a given candidate $y$ will be generated is evaluated using an Exponentiated \citet{luce1977choice} choice rule. Candidates with greater values of $a$ are more likely to be generated than candidates with smaller values:
% 
\begin{equation}
p(y) = \dfrac
    { \exp \{ \theta \cdot a(y, x) \} } 
    { \sum_i{ \exp \{ \theta \cdot a(y_i, x) \}  } }
    \label{eq:packer-choice}
\end{equation}
% 
where $\theta$ ($\theta \geq 0$) controls response determinism. 





\section{Snippets}


One of the core principles of PACKER (as well as its name) are inspired by earlier work in category learning \citep[see][]{hidaka2011packing}.


\clearpage

\section{Experiment 1}


\subsection{Participants \& Materials}

183 participants were recruited from Amazon Mechanical Turk. Participants were randomly assigned to one condition: 64 participants were assigned to the Cluster condition, 61 were assigned to the Row condition, and 58 were assigned to the XOR condition. Stimuli were squares varying in color (RGB 25--230) and side length (3.0--5.8cm). These stimuli are slight variants of those used by \cite{conaway2016similar}, see Figure \ref{fig:sample-stimuli} for samples. The assignment of perceptual features (color, size) to axes of the domain space (x, y) was counterbalanced across participants.

\begin{figure}
    \begin{center}
    \inputpgf{figs/}{stimuli-samples.pgf}
    \caption{Sample stimuli used in Experiments 1 and 2.}
    \label{fig:sample-stimuli}
    \end{center}
\end{figure}

\subsection{Procedure}

Participants began the experiment with a short training phase (3 blocks of 4 trials), where they observed exemplars belonging to the `Alpha' category. Participants were instructed to learn as much as they can about the Alpha category, and that they would answer a series of test questions afterwards. On each trial, a single Alpha category exemplar was presented, and participants were given as much time as they desired before moving on. Each block consisted of a single presentation of each of the members of the Alpha category, in a random order. Participants were shown the range of possible colors and sizes prior to training.

Following the training phase, participants were asked to generate four examples belonging to another category called `Beta'. As in \citet{jern2013probabilistic}, generation was completed using a sliding-scale interface. Two scales controlled the features (color, size) of the generated example. An on-screen preview of the example updated whenever one of the features was changed. Participants could generate any example along an evenly-spaced 9x9 grid, except for any previously generated Beta exemplars. Neither the members of the Alpha category nor the previously generated Beta examples were visible during generation. Prior to beginning the generation phase, participants read the following instructions:

\begin{displayquote}
As it turns out, there is another category of geometric figures called "Beta". Instead of showing you examples of the Beta category, we would like to know what you think is likely to be in the Beta category. 

You will now be given the chance to create examples of any size or color in order to show what you expect about the Beta category. You will be asked to produce 4 Beta examples - they can be quite similar or quite different to each other, depending on what you think makes the most sense for the category.

Each example needs to be unique, but the computer will let you know if you accidentally create a repeat.
\end{displayquote}

Following generation, participants completed a generalization phase wherein they classified novel examples into the Alpha and Beta categories without feedback. On each trial, a single example was presented, and participants were asked to classify it by clicking buttons labeled ``Alpha'' or ``Beta''. Participants classified a total of 81 items sampled along a 9x9 grid, including the members of the Alpha and Beta categories (randomly intermixed). These data were, however, collected to address a separate set of questions, and we do not discuss them in this paper.




\clearpage

\section{Experiment 2}


\subsection{Participants \& Materials}

122 participants were recruited from Amazon Mechanical Turk. 61 participants were randomly assigned to the Middle and Bottom conditions each. The stimuli were exactly as in Experiment 1. Again, the assignment of perceptual features (color, size) to axes of the domain space (x, y) was counterbalanced across participants.

\subsection{Procedure}

The procedure was exactly as in Experiment 1: participants first completed a short training phase, followed by the generation phase, followed by the generalization phase (data from this phase is not discussed in this report).







\section{Acknowledgments}
A previous version of this work appeared in the Proceedings of the Thirty-Ninth Annual Conference of the Cognitive Science Society. Support for this research was provided by the Office of the VCRGE at the UW - Madison with funding from the WARF. We thank Alan Jern and Charles Kemp for providing code and data.



\end{flushleft}


% references section
\clearpage
\bibliographystyle{apacite}
\bibliography{citations.bib}
\clearpage



\end{document}
