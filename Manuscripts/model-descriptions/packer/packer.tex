%!TEX output_directory = latex_out/

\documentclass[12pt]{article}
\usepackage[letterpaper, margin=1in, headheight=15pt]{geometry}
\usepackage{amsmath, listings}


\begin{document}

\section{The PACKER Model: Producing Alike or Contrasting Knowledge with Exemplar Representations}


This model assumes that people represent categories as exemplars in a multidimensional space, and that generation is constrained by both similarity to members of the category being generated \textit{and} dissimilarity to members of other categories. The model assumes people generate categories that are dissimilar to known categories and have strong within-class similarity.

The similarity between exemplars $x_i$ and $x_j$ is computed using Shepard's law:

\begin{equation}
s\left(x_i,x_j\right) = \exp \left\{ -c \left[\sum_{k}{ w_k \left| x_{ik} - x_{jk} \right|^r }\right]^{1/r} \right\}
\label{eq:similarity}
\end{equation}

When prompted to make a generation decision, participants are thought to consider both similarity to examples from other categories as well as similarity to examples in the target category. More formally, the aggregated similarity $a$ between candidate $y$ and the model's stored exemplars $x$ can be computed as:

\begin{equation}
  a(y, x) = \sum_j{f(x_j) s(y, x_j)}
\end{equation}

Where $f(x_j)$ is a function specifying each stored example's degree of contribution toward generation. Although $f(x_j)$ may be set arbitrarily, in PACKER it is set according to class assignment. For known members of the target category, $f(x_j) = \gamma$. For members of contrast categories, $f(x_j) = -1 + \gamma$. $\gamma$ is thus a free parameter ($0 \leq \gamma \leq 1$) controlling the trade-off between within-class similarity and between-class dissimilarity: $\gamma = 1$ produces exclusive consideration of same-category members, and $\gamma = 0$ produces exclusive consideration of opposite-category members. When $\gamma = 0.5$, the similarity to contrast categories is effectively subtracted from the similarity to the target category.

The probability that a given item $y$ will be generated given the model's memory $x$ is computed using relative summed similarity values across all generation candidates $y_i$:

\begin{equation}
p(y) = \dfrac
{ \exp  \left \{ \theta \cdot a \left( y, x \right) \right \} } 
{ \sum_i{ \exp  \left \{ \theta \cdot a \left( y_i, x \right) \right\}  } }
\label{eq:packer-choice}
\end{equation}

Where $\theta$ ($\geq 0$) is a free parameter controlling overall response determinism.



\end{document}








