%!TEX output_directory = latex_out/

\documentclass[12pt]{article}
\usepackage[letterpaper, margin=1in, headheight=15pt]{geometry}
\usepackage{amsmath,fancyvrb}



\begin{document}
\VerbatimFootnotes

\section*{Conjugate Implementation of the Jern \& Kemp (2013) Model with Representativeness}

The representativeness model is very similar to the Jern \& Kemp (2013)
hierarchical Bayesian model. The essential difference here is the ultimate
formulation of response probabilities. Where in the original hierarchical
Bayesian model the exemplars are drawn from Gaussian distributions, in the
representativeness model each exemplar is drawn in proportion to the exemplar's
representativeness. The representativeness of an exemplar is defined by
Tenenbaum and Griffiths (2012) as the relative evidence that is provided by the
exemplar $x$ for a given hypothesis $h$ compared to all other hypotheses $h'$:

\begin{equation}
  R(x,h) = \log \dfrac{p(x|h)}{\sum_{h'\neq h}{p(x|h')p(h')}}
\label{representativeness}
\end{equation}

For simplicity, and consistency with the original hierarchical Bayesian model,
we define $h_{C}$ as a multivariate normal distribution for a particular
category $C$ parameterised by $\mu_C$ and $\Sigma_C$. For completeness, this
document will repeat some information from the hierarchical Bayesian model
document and describee how we compute these variables.

\subsection*{Computing $\mu_C$}
Assuming $(\mu_C, \Sigma_C)$ are Normal-Inverse-Wishart distributed (unknown mean, unknown variance):
\
\begin{equation}
  \mu_C = \dfrac
    {\kappa\mu_{0} + n_C \bar{x_C}}
    {\kappa + n_C}
    \label{eq:category_mus}
\end{equation}
\ 
where:
\begin{itemize}
    \setlength\itemsep{-0.5em}
    \item $\mu_{0}$ is the prior mean along $p$ dimensions. Here we set it to the middle of the space.
    \item $\kappa$ is a scalar hyper-parameter, roughly weighting the importance of $\mu_{0}$. $\kappa$ must be greater than zero.
    \item $n_C$ is the number of observations in $x_C$
    \item $\bar{x_C}$ is the sample mean along $p$ dimensions
\end{itemize}

In the case of a populated class, $\mu_{C}$ ends up lying somewhere between $\mu_{0}$ and $\bar{x_C}$, depending on $\kappa_{0}$ and $n_C$. In the case of an empty class, $n_C = 0$, Equation \ref{eq:category_mus} reduces to $\mu_{C} = \mu_{0}$. Because we set $\mu_0$ to the center of the space, this outcome is the same as if we had integrated over all possible $\mu_C$. 

In practice, if $n_C = 0$, the model picks a stimulus at random from all candidates (uniform probabilities). 

\subsection*{Computing $\Sigma_D$}

Unlike $\mu_C$, $\Sigma_C$ cannot be computed considering only the members of category $y$. Instead, $\Sigma_C$ is influenced both by the distribution of $x_C$ and by members of other categories through $\Sigma_D$.

$\Sigma_D$ is inferred based on the observed (empirical) category covariances $C_y$. We assume these covariances to be Wishart-distributed, and so $\Sigma_D$ can be computed as:

\
\begin{equation}
    \Sigma_D = \Sigma_0 + \sum_{C}{C_C}
\end{equation}
\
$\Sigma_{0}$ is a $d$-by-$d$ prior covariance matrix. We use a $d$-dimensional identity matrix $I_d$ multiplied element-wise against a free parameter, $\lambda$, controlling the amount of variance assumed by the prior:
\
\begin{equation}
    \Sigma_0 =  \lambda I_d
\end{equation}

Thus, categories are assumed to have some degree of variance along each feature (specified by $\lambda$), but not are assumed to possess feature-feature correlations. Differences in the assumed variance among the features, similarly to weighting in an exemplar model, can be implemented through a small change to the equation. Specifically, the variance assumed of dimension $k$ is given by:
\
\begin{equation}
    \Sigma_{0k} =  \lambda w_k d
\end{equation}
\
where $d$ is the number of dimensions, and $w_k$ ($0 \leq w_k \geq 1$, $\sum_k{w_k} = 1$) indicates the dimension's relative share of the total assumed variability. Under this system, evenly distributed weights result in uniformly assumed variances, equal to $\lambda$.

\subsection*{Computing $\Sigma_C$}

Assuming $(\mu_C, \Sigma_C)$ are Normal-Inverse-Wishart distributed, $\Sigma_C$ can be computed as:

\begin{equation}
  \Sigma_C = [\Sigma_D\nu + C_C +
    \dfrac
    {\kappa n_C}
    {\kappa + n_C}
    (\bar{x_C}-\mu_C)(\bar{x_C}-\mu_C)^T
  ] (\nu + n_C)^{-1}
  \label{eq:Sigma_y}
\end{equation}
\
$\kappa$, $\bar{x_C}$, $C_C$, $n_C$, $\mu_0$,  are the same values as described above. $\nu$ is an additional free parameter, weighting the importance of $\Sigma_{D}$. $\nu$ must be greater than $d-1$. When $x_b$ is empty, Equation \ref{eq:Sigma_y} reduces to $\Sigma_C = \Sigma_D$.

\subsection*{Computing response probabilities $p(y | x_C)$}

As mentioned earlier, this is the point where the representativeness model
diverges from the hierarchical Bayesian model. Specifically, the probability of
generating exemplar $x$ is proportional to its representativeness:

\begin{equation}
  p(x) \propto R(x,h) ,
\end{equation}
where $h$ is a multivariate normal distribution parameterised by $\mu_{C}$ and
$\Sigma_{C}$.
 
In practice, $p(x)$ is computed by first obtaining the representativeness of
every possible generation candidate $x_i$. The end probability is a
normalization of these values:

\begin{equation}
  p(x) = \dfrac
    {\exp( \theta \cdot R(x,h))}
    {\sum_i \exp( \theta \cdot R(x_i,h_i))} 
\end{equation}
\
where $\theta$ is a response determinism parameter.

\subsection*{Description of free parameters}

\begin{itemize}
    \setlength\itemsep{-0.5em}
    \item $\kappa$. Scalar, $\kappa>0$. Weights the importance of $\mu_{0}$ in inferring category $\mu_{C}$.
    \item $\lambda$. Scalar, $\lambda>0$. Sets the assumed variance in the domain prior, $\Sigma_{0}$.
    \item $\nu$. Scalar, $\nu > d-1$. Weights the importance of $\Sigma_{D}$ in inferring the domain $\Sigma_{C}$.
    \item $\theta$. Scalar, $\theta > 0$. Response determinism parameter.
\end{itemize}



\end{document}
