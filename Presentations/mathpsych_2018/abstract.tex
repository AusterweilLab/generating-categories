Research into categorization has primarily focused on how people learn
categories and extend category membership to novel exemplars. However, it is
unclear how people generate new categories. In previous work, we presented
PACKER (\emph{Producing Alike and Contrasting Knowledge using Exemplar
Representations}), an exemplar model that accounts for category generation by
measuring not only the similarity of an exemplar with a target category but also
its dissimilarity with contrast categories. As predicted by PACKER, we found
that people generated new categories that contrast from a given category. In
this work, we examine whether category contrast also plays a role in category
learning. Specifically, we conducted a novel experiment where participants
learned categories generated by participants in our previous work. We
demonstrate how category contrast allows PACKER to explain empirical data in
both category learning and category generation paradigms better than competing
theories, providing evidence for category contrast as a fundamental factor in
categorization.